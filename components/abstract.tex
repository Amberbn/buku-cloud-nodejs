\begin{abstract}
\thispagestyle{plain}
\setcounter{page}{1}
\addcontentsline{toc}{chapter}{\numberline{}Kata Pengantar}

Selamat datang pada workshop Pengambangan Aplikasi Cloud Computing Menggunakan Node.js. Workshop ini merupakan workshop yang dirancang untuk keperluan memberikan pengetahuan mendasar pengembangan aplikasi berbasis Cloud Computing. Untuk mengikuti materi yang ada pada workshop ini, peserta diwajibkan membawa laptop yang telah terisi berbagai software infrastruktur yang diperlukan dan peserta diwajibkan mengetahui penggunaan software-software tersebut:
\begin{itemize}
  \item Sistem operasi Linux (distribusi apa saja)
  \item Git (untuk SCM)
  \item Ruby dan Rubygem
  \item mongoDB (basis data NOSQL)
  \item Vim (untuk mengedit source code)
\end{itemize}

Software utama untuk keperluan workshop ini yaitu Node.js serta command line tools dari provider Cloud Computing (materi ini menggunakan fasilitas dari CloudFoundry), akan dibahas pada workshop. Materi akan lebih banyak berorientasi ke command line / shell sehingga para peserta sebaiknya sudah memahami cara-cara menggunakan shell di Linux.

Have fun!
\end{abstract}

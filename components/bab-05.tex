\chapter{Mengelola Paket Menggunakan npm}

\section{Apakah npm Itu?}

Node.js memungkinkan developer untuk mengembangkan aplikasi secara modular dengan memisahkan berbagai \textit{resusable code} ke dalam pustaka. Berbagai pustaka tersebut bisa diperoleh di \url{http://npmjs.org}. Node.js menyediakan perintah \textit{npm} untuk mengelola paket pustaka di repositori tersebut.

\section{Menggunakan npm}

\subsection{Instalasi Paket}

\subsection{Struktur Instalasi Paket Node.js}

Dalam installasi paket pustaka, berkas-berkas akan terletak dalam folder lokal aplikasi \textite{node_modules}. Pada mode installasi paket pustaka global (dengan -g atau --global dibelakang baris perintah), paket pustaka akan dipasang pada \textite{/usr/lib/node_modules} (dengan lokasi installasi Node.js standar). Mode global memungkinkan paket pustaka digunakan tanpa memasang paket pustaka pada setiap folder lokal aplikasi. Mode global ini juga membutuhkan hak administrasi lebih (sudo atau root) dari pengguna agar dapat menulis pada lokasi standar.

Daftar paket pustaka yang terpasang dapat dilihat menggunakan perintah berikut :

\lstset{language=bash,caption=Hasil instalasi Flatiron}
\begin{lstlisting}
$ npm ls
\end{lstlisting}

Selain melihat daftar paket pustaka yang digunakan dalam aplikasi maupun global, perintah diatas juga akan menampilkan paket depedensi dalam struktur pohon. Berikut contoh struktur installasi dari paket pustaka lokal aplikasi :

\lstset{language=bash,caption=Hasil instalasi Flatiron}
\begin{lstlisting}
$ npm ls
bpdp-m1-hello@0.0.1 /home/adzy/Projects/buku-cloud-nodejs/src/modul-1/hello
└─┬ express@3.0.0 
  ├── commander@0.6.1 
  ├─┬ connect@2.6.0 
  │ ├── bytes@0.1.0 
  │ ├── formidable@1.0.11 
  │ ├── pause@0.0.1 
  │ ├── qs@0.5.1 
  │ └─┬ send@0.0.4 
  │   └── mime@1.2.6 
  ├── cookie@0.0.4 
  ├── crc@0.2.0 
  ├── debug@0.7.0 
  ├── fresh@0.1.0 
  ├── methods@0.0.1 
  ├── mkdirp@0.3.3 
  ├── range-parser@0.0.4 
  └─┬ send@0.1.0 
    └── mime@1.2.6 
\end{lstlisting}

\subsection{Menghapus Paket / Uninstall}

\subsection{Berbagai Perintah Penting Pengelolaan Paket dengan npm}


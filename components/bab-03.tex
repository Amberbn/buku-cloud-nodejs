\chapter{Paradigma Pemrograman di JavaScript}

\section{Pemrograman Fungsional}

Pemrograman fungsional, atau sering disebut \textit{functional programming}, selama ini lebih sering dibicarakan di level para akademisi. Meskipun demikian, saat ini terdapat kecenderungan paradigma ini semakin banyak digunakan di industri. Contoh nyata dari implementasi paradigma ini di industri antara lain adalah Scala (\url{http://www.scala-lang.org}), OCaml (\url{http://www.ocaml.org}), Haskell (\url{http://www.haskell.org}), Microsoft F\# (\url{http://fsharp.org}), dan lain-lain. Dalam konteks paradigma pemrograman, peranti lunak yang dibangun menggunakan pendekatan paradigma ini akan terdiri atas berbagai fungsi yang mirip dengan fungsi matematis. Fungsi matematis tersebut di-evaluasi dengan penekanan pada penghindaran \textit{state} serta \textit{mutable data}. Bandingkan dengan paradigma pemrograman prosedural yang menekankan pada \textit{immutable data} dan definisi berbagai prosedur dan fungsi untuk mengubah \textit{state} serta data.

JavaScript bukan merupakan bahasa pemrograman fungsional yang murni, tetapi ada banyak fitur dari pemrograman fungsional yang terdapat dalam JavaScript. Dalam hal ini, JavaScript banyak dipengaruhi oleh bahasa pemrograman Scheme (\url{http://www.schemers.org/}). Bab ini akan membahas beberapa fitur pemrograman fungsional di JavaScript. Pembahasan ini didasari pembahasan di bab sebelumnya tentang Fungsi di JavaScript.

\subsection{Ekspresi Lambda}

\index{Ekspresi Lambda}Ekspresi lambda / \textit{lambda expression} merupakan hasil karya dari ALonzo Church sekitar tahun 1930-an. Aplikasi dari konsep ini di dalam pemrograman adalah penggunaan fungsi sebagai parameter untuk suatu fungsi. Dalam pemrograman, \textit{lambda function} sering juga disebut sebagai fungsi anonimus (fungsi yang dipanggil/dieksekusi tanpa ditautkan (\textit{bound}) ke suatu \textit{identifier}). Berikut adalah implementasi dari konsep ini di JavaSCript:

\lstset{language=JavaScript,caption=Ekspresi Lambda di JavaScript}
\lstinputlisting{src/bab-03/lambda.js}

\subsection{Higher-order Function}

\index{Higher-order Function}\textit{Higher-order function} (sering disebut juga sebagai \textit{functor} adalah suatu fungsi yang setidak-tidaknya menggunakan satu atau lebih fungsi lain sebagai parameter dari fungsi, atau menghasilkan fungsi sebagai nilai kembalian. 

\lstset{language=JavaScript,caption=Higher-order Function di JavaScript}
\lstinputlisting{src/bab-03/hof.js}

\subsection{Closure}

\index{Closure}Suatu \textit{closure} merupakan definisi suatu fungsi bersama-sama dengan lingkungannya. Lingkungan tersebut terdiri atas fungsi internal serta berbagai variabel lokal yang masih tetap tersedia saat fungsi utama / closure tersebut selesai dieksekusi. 

\lstset{language=JavaScript,caption=Closure di JavaScript}
\lstinputlisting{src/bab-03/closure.js}

\subsection{Currying}

\index{Currying}\textit{Currying} memungkinkan pemrogram untuk membuat suatu fungsi dengan cara menggunakan fungsi yang sudah tersedia secara parsial, artinya tidak perlu menggunakan semua argumen dari fungsi yang sudah tersedia tersebut.

\lstset{language=JavaScript,caption=Currying di JavaScript}
\lstinputlisting{src/bab-03/currying.js}

\section{Pemrograman Berorientasi Obyek}

\subsection{Pengertian}

\index{PBO}Pemrograman Berorientasi Obyek (selanjutnya akan disingkat PBO) adalah suatu paradigma pemrograman yang memandang bahwa pemecahan masalah pemrograman akan dilakukan melalui definisi berbagai kelas kemudian membuat berbagai obyek berdasarkan kelas yng dibuat tersebut dan setelah itu mendefinisikan interaksi antar obyek tersebut dalam memecahkan masalah pemrograman. Obyek bisa saling berinteraksi karena setiap obyek mempunyai properti (sifat / karakteristik) dan \textit{method} untuk mengerjakan suatu pekerjaan tertentu. Jadi, bisa dikatakan bahwa paradigma ini menggunakan cara pandang yang manusiawi dalam penyelesaian masalah.

Dengan demikian, inti dari PBO sebenarnya terletak pada kemampuan untuk mengabstraksikan berbagai obyek ke dalam kelas (yang terdiri atas properti serta method). Paradigma PBO biasanya juga mencakup \textit{inheritance} atau pewarisan (sehingga terbentuk skema yang terdiri atas \textit{superclass} dan \textit{subclass}). Ciri lainnya adalah \textit{polymorphism} dan \textit{encapsulation} / pengkapsulan.

JavaScript adalah bahasa pemrograman yang mendukung PBO dan merupakan implementasi dari ECMAScript. Implementasi PBO di JavaScript adalah \textit{prototype-based programming} yang merupakan salah satu subset dari PBO. Pada \textit{prototype-based programming}, kelas / \textit{class} tidak ada. Pewarisan diimplementasikan melalui \textit{prototype}.

\subsection{Definisi Obyek}

\index{Obyek}Definisi obyek dilakukan dengan menggunakan definisi \textit{function}, sementara \textit{this} digunakan di dalam definisi untuk menunjukkan ke obyek tersebut. Sementara itu, Kelas.prototype.namaMethod digunakan untuk mendefinisikan method dengan nama method namaMethod pada kelas Kelas. Perhatikan contoh pada listing berikut.

\lstset{language=JavaScript,caption=Definisi obyek di JavaScript}
\lstinputlisting{src/bab-03/obyek.js}

\subsection{\textit{Inheritance} / Pewarisan}

\index{Pewarisan}Pewarisan di JavaScript bisa dicapai menggunakan \textit{prototype}. Listing program berikut memperlihatkan bagaimana pewarisan diimplementasikan di JavaScript.

\lstset{language=JavaScript,caption=Pewarisan di PBO JavaScript}
\lstinputlisting{src/bab-03/inheritance.js}

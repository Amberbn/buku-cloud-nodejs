\chapter{REPL dan Dasar-dasar JavaScript di Node.js}

\section{REPL}

REPL adalah lingkungan pemrograman interaktif, tempat developer bisa mengetikkan program per baris dan langsung mengeksekusi hasilnya. Biasanya ini digunakan untuk menguji perintah-perintah yang cukup dijalankan pada satu baris atau satu blok segmen kode sumber saja. Karena fungsinya itu, maka istilah yang digunakan adalah REPL (read-eval-print-loop), yaitu loop atau perulangan baca perintah - evaluasi perintah - tampilkan hasil. REPL sering juga disebut sebagai \textit{interactive top level} atau \textit{language shell}. ``Tradisi'' ini sudah dimulai sejak jaman LISP di mesin UNIX di era awal pengembangan \textit{development tools}. Saat ini hampir semua \textit{interpreter/compiler} mempunyai REPL, misalnya Python, Ruby, Scala, PHP, berbagai interpreter/compiler LISP, dan tidak ketinggalan Node.js. 

\subsection{Mengaktifkan REPL}

Untuk mengaktifkan REPL dari Node.js, \textit{executable command line program}-nya adalah \textbf{node}. Jika \textbf{node} dipanggil dengan argumen nama file JavaScript, maka file JavaScript tersebut akan dieksekusi, sementara jika tanpa argumen, akan masuke ke REPL:

\lstset{language=bash,caption=Node.js REPL}
\begin{lstlisting}
$ node
> .help
.break	Sometimes you get stuck, this gets you out
.clear	Alias for .break
.exit	Exit the repl
.help	Show repl options
.load	Load JS from a file into the REPL session
.save	Save all evaluated commands in this REPL session to a file
>
\end{lstlisting}

Tanda ``\textbf{>}'' adalah tanda bahwa REPL Node.js siap untuk menerima perintah. Untuk melihat perintah-perintah REPL, bisa digunakan \textbf{.help}.

\subsection{Perintah-perintah REPL}

Pada sesi REPL, kita bisa memberikan perintah internal REPL maupun perintah-perintah lain yang sesuai dan dikenali sebagai perintah JavaScript. Perintah internal REPL Node.js terdiri atas:
\begin{itemize}
  \item \textbf{.break}: keluar dan melepaskan diri dari "keruwetan" baris perintah di REPL.
  \item \textbf{.clear}: alias untuk .break
  \item \textbf{.exit}: keluar dari sesi REPL (bisa juga dengan menggunakan Ctrl-D)
  \item \textbf{.help}: menampilkan pertolong perintah internal REPL
  \item \textbf{.load}: membaca dan mengeksekusi perintah-perintah JavaScript yang terdapat pada suatu file.
  \item \textbf{.save}: menyimpan sesi REPL ke dalam suatu file.
\end{itemize}

Contoh untuk \textbf{.load}:

\lstset{language=JavaScript,caption=Contoh penggunaan .load dalam REPL}
\begin{lstlisting}
> .load /home/bpdp/kerjaan/src/javascript/nodejs/server.js
> var http = require('http');
undefined
> http.createServer(function (req, res) {
...     res.writeHead(200, {'Content-Type': 'text/plain'});
...       res.end('Hello, world!\n');
... }).listen(8124, "127.0.0.1");
{ _connections: 0,
  connections: [Getter/Setter],
  allowHalfOpen: true,
  _handle: null,
  _events: 
   { request: [Function],
     connection: [Function: connectionListener] },
  httpAllowHalfOpen: false }
> console.log('Server running at http://127.0.0.1:8124/');
Server running at http://127.0.0.1:8124/
undefined
>
\end{lstlisting}

Setelah keluar dari sesi REPL, maka port akan ditutup dan hasil eksekusi di atas akan dibatalkan. 

Untuk menyimpan hasil sesi REPL menggunakan \textbf{.save}, jika tanpa menyebutkan direktori, maka akan disimpan di direktori aktif saat itu. Contoh:
\lstset{language=bash,caption=Contoh penggunaan perintah .save di sesi REPL}
\begin{lstlisting}
$ node
> console.log("Selamat datang di Node.js")
Selamat datang di Node.js
undefined
> .save /home/bpdp/kerjaan/src/javascript/nodejs/hello.js
Session saved to:/home/bpdp/kerjaan/src/javascript/nodejs/hello.js
> [bpdp@bpdp-arch ~]$ cat /home/bpdp/kerjaan/src/javascript/nodejs/hello.js 
console.log("Selamat datang di Node.js")
\end{lstlisting}

\section{Dasar-dasar JavaScript di Node.js}

\subsection{Nilai/Values}


\subsection{Variabel}


\subsection{Aliran Kendali}

\subsection{Fungsi}

\subsection{Penanganan Error}

\subsection{Struktur Data dan Representasi JSON}

